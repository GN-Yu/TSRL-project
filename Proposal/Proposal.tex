\documentclass[12pt,letterpaper,oneside]{article}
\usepackage[utf8]{inputenc}
\usepackage{amsmath,amsthm,amssymb,amsfonts}
\usepackage{graphicx}
\usepackage{breqn}
\usepackage{extarrows}
\usepackage{hyperref}
\usepackage{ulem}
\usepackage{bbm}

\usepackage{geometry}
\geometry{left=2.5cm, right=2.5cm, top=2.5cm, bottom=2.5cm}

%\usepackage{fancyhdr}
%\pagestyle{fancy}
%\fancyhf{}
%\renewcommand\headrulewidth{0.7pt} %initially 0.4pt
%\cfoot{\thepage}
%\fancyhead[C]{TSRL Interview}
%\fancyhead[L]{Question 3 Solutions}
%\fancyhead[R]{Guoning Yu}

%\usepackage{mathtools}
%\usepackage{ntheorem}
\usepackage{enumerate}
%%\usepackage[ruled]{algorithm2e}
%
%\usepackage{booktabs}

\title{Project Proposal for TSRL Interview Question 3}
\author{Guoning Yu}
\date{Winter 2023}

%\bibliographystyle{plain}

\begin{document}
	
\baselineskip 0.6cm
\maketitle

\section{Introduction}
	This question is about reading the paper \textit{Temporal Difference Learning for High-Dimensional PIDEs with Jumps, Liwei Lu, Hailong Guo, Xu Yang, Yi Zhu, 2023} and replicate the reinforcement learning algorithm in it. 
	
	A \textit{Partial Integro-Differential Equation (PIDE)} is a type of equation that combines features of both partial differential equations (PDEs) and integral equations. In a PIDE, the unknown function is defined over a domain and the equation involves both partial derivatives and integral operators acting on this function. More specifically, a PIDE typically takes the  general form
	\begin{equation}
		\frac{\partial u}{\partial t}+\mathcal{L} u+\mathcal{I} u=f,
	\end{equation}
	in which $u = u(\textbf{x}, t)$ is the unknown function, $\mathcal{L} u$ denotes a differential operator, which is often a linear or nonlinear combination of spatial derivatives of $u$, and $\mathcal{I} u$ represents an integral operator, involving an integral of $u$ with respect to space, possibly weighted by a kernel function.
	
	\textit{Jumps} refer to sudden, significant changes in the price of an asset. These can be caused by various events such as economic announcements, geopolitical events, or sudden market movements. Unlike the normal price fluctuations assumed in models like Black-Scholes, jumps represent discontinuities, where the asset price changes significantly in a very short period, often unpredictably. 
		
	PIDEs are particularly common in financial mathematics for modeling options pricing in markets with jumps or other non-continuous features, as they can capture both the local changes (via the differential part) and the global or aggregate effects (via the integral part), which addresses the limitations of models like Black-Scholes in markets with jumps.
	
	In the context of PIDEs, the integral part, namely, the ``non-local'' term typically has the general form
	$$
	\mathcal{I}(u)(t, x)=\int_{\Omega} K(x, y, t) u(y, t) d y,
	$$
	where $\Omega$ is the domain of integration, and $K(x, y, t)$ is a kernel function that specifies how values of $u$ at different points $y$ contribute to the integral at the point $x$. The kernel often encapsulates the nature of the non-local interactions.
	
	However, solving PIDEs is extremely hard in high-dimensional cases given a large number of underlying assets. There are previous work using deep neural networks (NN) to solve PDEs, while the results on solving PIDEs with NN method are not as extensive.
	
	The paper is about solving PIDEs numerically. A group of L\'evy-type forward-backward stochastic processes is introduced based on the target PIDE. It gives an algorithm that ... They used a method of Temporal Difference Learning under the Reinforcement Learning framework.  The result presented in the paper is ... The significance of their algorithm is ...

	
\section{Typos and Errors in Paper}
	The typos and errors in the paper are listed as follows.
	\begin{enumerate}
		\item In conclusion section: "the future work" should be "future work".
	\end{enumerate}


\section{Unclear Parts}
	The parts that I had hard time understanding are:
	
	\begin{enumerate}
		\item 
	\end{enumerate}

\section{Replication Results}
	I used TensorFlow to replicate ... The project implementation is in the \href{https://github.com/GN-Yu/TSRL-project}{GitHub Repo}.
	
	In my replication, I used the following techniques.
	
	The results of my replication are ... 
	
	Comparing with the paper's original results, ....
	
	
	
\section{Notes for Some Difficult Points}

	Here are some difficult points that might needs extra attention for people unfamiliar with ...
	
	\begin{enumerate}
		\item 
	\end{enumerate}
	
\end{document}